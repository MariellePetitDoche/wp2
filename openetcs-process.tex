
The development process for openETCS shall closely follow the V-Model process
presented in CENELEC EN 50128 Figure~4. It is separated into different distinct
phases which allow for iteration in the phases and between the phases. The
participating roles in the software development process are defined in CENELEC
EN 50128 Figure~2.

\subsection{Participating Roles}
\label{sec:participating-roles}

Each of the participants must be noted and her name and role recorded. Which
role can be taken by the same person is regulated as shown in
Figure~\ref{fig:preferred-roles}.

\subsubsection{Requirements Maganger (RQM)}
\label{sec:requ-magang-rqm}

The main task of the RQM is to take care of the software requirements. She will
specify the software requirements, ensure their traceability wrt. system level
requirements and ensure the consistency of the SW requirement specification.

\subsubsection{Designer (DES)}
\label{sec:designer}

The main task of the DES is to create acceptable solutions from the SW
requirement specifications. She is responsible for the decisions which design
methods to apply and which tools to use. The DES develops the software component
specifications and ensures their traceability and consistency wrt. SW
requirements specification.

\subsubsection{Implementer (IMP)}
\label{sec:implementer}

The main task of the IMP is the transformation of the software design solution
into source code, DSLs or other appropriate formalisms. Besides the
implementation she will produce the documentation describing the implementation,
the applied methods, data types and data structures. She will also ensure the
traceability wrt. SW design. The IMP will apply the specified coding standards,
use the specified programming languages and use safety design principles.

\subsubsection{Tester (TST)}
\label{sec:tester}

The TST is responsible for the planning of all test activities and will produce
the test specification which includes the test cases and their objectives. She
is responsible to assure the execution of the SW tests, the recording of their
outcomes and to select the test equipments and methods. The TST will communicate
relevant deviations from the specifications to the change management.

\subsubsection{Verifier (VER)}
\label{sec:verifier}

The VER is responsible to produce the SW verification plan which specifies what
and how it has to be verified and what has to be produced as evidence. She will
evaluate all deviations from the verification plan according to their potential
impact and risk and communicate all deviations to the change management. The VER
records all outcomes of the verification activities and will produce the
verification report.

\subsubsection{Integrator (INT)}
\label{sec:integrator}

The main task of the INT is the test of the integration developed SW into the
target HW based on the design specification. She will specify the necessary
integration sequence, the necessary input components and the expected resulting
components. The INT will record any deviation from the integration test
specification, communicate them the change management and produce a system
integration report of the overall outcome.

\subsubsection{Validator (VAL)}
\label{sec:validator}

The main task of the VAL is to ensure that the software requirements meet the
intended usage in the specified application area and environment. For this she
will develop an understanding of the system and its intended environment. She
will then review the developed SW against the SW specification, evaluate the
conformity of the development process to the standard and the assigned SIL. She
will review the correctness, consistency and adequacy of the verification,
testing, test cases and executed tests, as well as ensuring that the validation
plan was carried out completely.

Based on these analyses she will review the deviations, classify them in terms
of risk, communicate them to the change management and will give a
recommendation on the suitability of the developed SW for the intended task, as
well as indicate necessary constraints.

She will audit and review the overall project wrt. generic development process,
in particular verify the traceability of all SW requirements. She will ensure
that all remaining non-conformities are resolved directly or by risk control /
transfer measures. Finally the VAL will produce a validation report and give her
agreement or disagreement for the release of the SW. The produced documents will
be transmitted to the assessor.

\subsubsection{Assessor (ASR)}
\label{sec:assessor}

The main task of the ASR is to develop an assessment plan which is communicated
to the safety authority and the client. The ASR will develop an understanding of
the system and its intended environment and will evaluate the competency of the
project staff and the organization, the verification and validation with their
supporting evidence, the quality management of the development process and the
configuration and change management with its evidence of use and application.

She will identify and evaluate the risk of the deviations from the SW
specification, produce an assessment report and ensure that the assessment plan
is executed. She will carry out audits and inspections of the overall
development process and give a professional view on the fitness of the developed
SW for he intended use. Based on this she will produce an assessment report and
record there her findings of the assessment process.

The ASR must be independent of the developing organization, so this must be kept
in mind when appointing her.

\subsubsection{Project Manager (PM)}
\label{sec:project-manager}

The main task of the project manager is to ensure the independence of the roles
and organizations as specified in CENELEC EN 50128 (see
Figure~\ref{fig:preferred-roles}). She shall be responsible to allocate the
necessary resources, ensure the competency for the allocated rules and allow
sufficient time for the proper implementation of all required tasks. The PM
shall ensure that safety requirements of other stakeholders are met and is
responsible for the delivery and deployment of the SW. She will endorse safety
deliverables and will ensure sufficient records and the traceability of
requirements for safety related decisions.

\subsubsection{Configuration Manager (CM)}
\label{sec:conf-manag}

The main task of the CM is the responsibility for the SW configuration
management plan. She owns the configuration management system and ensures the
clear identification and independent versioning of the SW components. The CM
shall prepare Release Notes for the SW and document incompatible components if
applicable.

\subsection{Documents / Plans}
\label{sec:documents--plan}

The PM shall decide the allocation of the roles to the participants, according
to the requirements of the standard and the knowledge and competence of the
participants.

The allocated roles will then proceed to produce the required plans according to
CENELEC EN 50128:

\begin{itemize}
\item SW Quality Assurance Plan
\item SW Configuration Management Plan (CM)
\item SW Verification Plan (VER)
\item SW Validation Plan (VAL)
\item SW Maintenance Plan
\end{itemize}



\subsection{Lifecycle Phases}
\label{sec:lifecycle-phases}

The first task is to decide a development process, the recommended lifecycle
model in CENELEC EN 50128  is the V-model which shown in
Figure~\ref{fig:develop-lifecycle-cenelec}, it consists of the following phases:

\begin{itemize}
\item System Development Phase
\item SW Development Phases
  \begin{itemize}
  \item SW Requirements Phase
  \item SW Architecture and Design Phase
  \item SW Component Design Phase
  \end{itemize}
\item SW Component Implementation Phase
\item SW Test / Validation Phases
  \begin{itemize}
  \item SW Validation Phase
  \item SW Integration Phase
  \item SW Component Testing Phase
  \end{itemize}
\end{itemize}

Each of the SW development phases shall have an analogous test / validation
phase

\subsection{System Development Phase}
\label{sec:syst-devel-phase}

The system development phase is finished externally and consists the creation of
the ERTMS/ETCS specification as the SRS of Subset 26.

\subsection{SW Development Phase}
\label{sec:sw-development-phase}

The formalization of the informal requirements of the SRS will be part of the
openETCS project, in particular for the open Proof approach. Any problems found
in the SRS in the formalization and verification of the informal requirements
for the SW development phases will be documented and communicated {\bf
  whom?}. %TODO

\subsubsection{SW Requirements Phase}
\label{sec:sw-requ-phase}



\paragraph{Tasks}
\label{sec:tasks}
\begin{itemize}
\item Formalization of the system specification
\item Formalization of the safety requirements of the system
\item Specification of the necessary tests and formal verification tasks
\end{itemize}

\paragraph{Reports}
\label{sec:sw-req-report}
\begin{itemize}
\item SW requirements verification report
\end{itemize}

\subsubsection{SW Architecture and Design Phase}
\label{sec:sw-arch-design}

\paragraph{Tasks}
\label{sec:tasks-1}



%%% Local Variables:
%%% mode: latex
%%% TeX-master: "wp-2.2"
%%% End:
